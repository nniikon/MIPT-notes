\documentclass[a4paper]{article}
\usepackage[utf8]{inputenc}
\usepackage[russian,english]{babel}
\usepackage[T2A]{fontenc}
\usepackage[left=10mm, top=20mm, right=18mm, bottom=15mm, footskip=10mm]{geometry}
\usepackage{indentfirst}
\usepackage{amsmath,amssymb}
\usepackage[italicdiff]{physics}
\usepackage{graphicx}
\usepackage{multirow}
\usepackage{svg}
\graphicspath{{images/}}
\DeclareGraphicsExtensions{.pdf,.png,.jpg}
\usepackage{wrapfig}
\usepackage{caption}
\captionsetup[figure]{name=Рисунок}
\captionsetup[table]{name=Таблица}
\title{\underline{Определение $C_p / C_v$ методом адиабатического расширения газа}}
\author{Каспаров Николай, Б01-304}

\begin{document}

\maketitle
\begin{center}
\Large{\textbf{ }}
\end{center}

\subparagraph{Цель работы:}

Определить отношение $C_p / C_v$ для воздуха по измерению давления в стеклянном сосуде.

\subparagraph{В работе используются:}

Стелянный сосуд, U-образный жидкостный манометр, разиновая груша.

\section{Теоретическое введение}

Адиабатический процесс -- процесс, происходящий без теплообмена с окружающей средой.

Формула Майера гласит, что:

\begin{equation}
    C_p - C_v = R 
\end{equation}

Отношение этих теплоемкостей часто используется и называется показателем адиабаты

\begin{equation}
     \gamma = \frac{C_p}{C_v}
\end{equation}

Известно, что на каждую степень свободы, приходится энергия, равная $kT/2$, 
где $k = 1.38 \cdot 10^{-23} $ Дж/К - постоянная Больцмана. Значит:

\begin{equation}
    U = \frac{i}{2}kN_AT = \frac{i}{2}RT \qquad \qquad
    C_v = \frac{i}{2}R
\end{equation}

Тогда, из формулы (2) получим:

\begin{equation}
     \gamma = \frac{i + 2}{i}
\end{equation}

В частности, для воздуха $\gamma_\text{в} = \frac{7}{5} = 1.4$.

В адиабатном процессе $\delta Q = 0$, тогда $dU = C_vdT$, получим $C_vdT + PdV = 0$.
Используя соотношение $PV = RT$, получим:

\begin{equation*}
    C_V \frac{dT}{T} + R \frac{dV}{V} = 0
\end{equation*}

Интегрируя, получим:

\begin{equation}
    PV^\gamma = const
\end{equation}

\subsection*{Экспериментальная установка}

В начале опыта мы, используя резиновую грушу, повышаем давление
в стеклянном сосуде до $P_1 > P_0$, через время температура сравняется с комнатной $T_1$.
Затем, откроем кран $K$, соединяющий сосуд с атмосферой, на малое $\Delta t$.
Через время $\Delta t_p$ давление сравняется с атмосферным,
а через $\Delta t_T$ температура сравняется с атмосферным. Очевидно, что:

\begin{equation}
    \Delta t_p \ll \Delta t \ll \Delta t_T
\end{equation}

Из уравнение адиабаты (5) получим:

\begin{equation}
    \biggl(\frac{P_1}{P_0}\biggl)^{\gamma-1} = \biggl(\frac{T_1}{T_2}\biggl)^\gamma, \text{ где}
\end{equation}

"1" - начальное состояние, 
"2" - момент выравнивания давления с атмосферным

Затем, газ снова сравнивает температуру с комнатной, $T_3 = T_1$.
В этом процессе также повышается и давление газа до $P_3$.

По закону Гей-Люссака получим:

\begin{equation}
    \frac{P_0}{T_2} = \frac{P_3}{T_3} = \frac{P_3}{T_1}
\end{equation}

Из уравнения (7), исключая температуры уравнением (8) получим:

\begin{equation*}
    \biggl(\frac{P_3}{P_0}\biggl)^{\gamma} = \biggl(\frac{P_1}{P_0}\biggl)^{\gamma - 1}
\end{equation*}

Выразим $\gamma$:

\begin{equation}
    \gamma = \frac{\ln(P_1/P_0)}{\ln(P_1/P_3)}
\end{equation}

Давления $P_1$ и $P_2$ отличаются от атмосферного
на величину гидростатического давления столба воды, который мы измеряем: 

\begin{equation}
    P_1 = P_0 + \rho gh_1 \qquad \qquad P_3 = P_0 + \rho gh_2
\end{equation}

Из уравнений (9) и (10), разложением логарифмов в ряд, получим:

\begin{equation}
    \gamma = \frac{\ln(1 + \rho gh_1 / P_0)}
                  {\ln(1 + \rho gh_1 / P_0) - \ln(1 + \rho gh_2 / P_0)} \approx 
             \frac{h_1}{h_1 - h_2}
\end{equation}

\section{Измерения}
                                                                                                                                                            
\includesvg[]{output}

\end{document}

