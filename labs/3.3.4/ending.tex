\section{Вывод}

В ходе данной лабораторной работы было подтверждено существование эффекта Холла в полупроводниках. Было получено значение константы Холла: $(3.038 \pm 0.014) \cdot 10^{-2} \ \frac{\text{м}^3}{\text{Кл}}$ и некотрых других параметров среды (подвижность, концентрация носителей заряда). Был экспериментально определен знак носителей заряда: отрицательный. Результаты вычислений удельного сопротивления и подвижности носителей зарядов не сходятся с табличными, что, вероятно, объясняется ошибкой при измерении разности потенциалов между точками 3 и 5. В свою очередь, графики описывают верные зависимости, возможно, произошла ошибка при фиксировании единиц измерения величин (например, вместо мкВ измерялись мВ), из-за чего последние из результатов отличаются от теоретических. Тем не менее концентрация носителей зарядов получилась схожей с той, что есть на самом деле.
